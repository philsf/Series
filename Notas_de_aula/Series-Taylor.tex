\everymath{\displaystyle}
%\documentclass[pdftex,a4paper]{article}
\documentclass[a4paper]{article}
%%classes: article, report, book, proc, amsproc

%%%%%%%%%%%%%%%%%%%%%%%%
%% Misc
% para acertar os acentos
\usepackage[brazilian]{babel} 
%\usepackage[portuguese]{babel} 
% \usepackage[english]{babel}
% \usepackage[T1]{fontenc}
% \usepackage[latin1]{inputenc}
\usepackage[utf8]{inputenc}
\usepackage{indentfirst}
\usepackage{fullpage}
% \usepackage{graphicx} %See PDF section
\usepackage{multicol}
\setlength{\columnseprule}{0.5pt}
\setlength{\columnsep}{20pt}
%%%%%%%%%%%%%%%%%%%%%%%%
%%%%%%%%%%%%%%%%%%%%%%%%
%% PDF support

\usepackage[pdftex]{color,graphicx}
% %% Hyper-refs
\usepackage[pdftex]{hyperref} % for printing
% \usepackage[pdftex,bookmarks,colorlinks]{hyperref} % for screen

%% \newif\ifPDF
%% \ifx\pdfoutput\undefined\PDFfalse
%% \else\ifnum\pdfoutput > 0\PDFtrue
%%      \else\PDFfalse
%%      \fi
%% \fi

%% \ifPDF
%%   \usepackage[T1]{fontenc}
%%   \usepackage{aeguill}
%%   \usepackage[pdftex]{graphicx,color}
%%   \usepackage[pdftex]{hyperref}
%% \else
%%   \usepackage[T1]{fontenc}
%%   \usepackage[dvips]{graphicx}
%%   \usepackage[dvips]{hyperref}
%% \fi

%%%%%%%%%%%%%%%%%%%%%%%%


%%%%%%%%%%%%%%%%%%%%%%%%
%% Math
\usepackage{amsmath,amsfonts,amssymb}
% para usar R de Real do jeito que o povo gosta
\usepackage{amsfonts} % \mathbb
% para usar as letras frescas como L de Espaco das Transf Lineares
% \usepackage{mathrsfs} % \mathscr

% Oferecer seno e tangente em pt, com os comandos usuais.
\providecommand{\sin}{} \renewcommand{\sin}{\hspace{2pt}\mathrm{sen}}
\providecommand{\tan}{} \renewcommand{\tan}{\hspace{2pt}\mathrm{tg}}

% dt of integrals = \ud t
\newcommand{\ud}{\mathrm{\ d}}
%%%%%%%%%%%%%%%%%%%%%%%%



\begin{document}

%%%%%%%%%%%%%%%%%%%%%%%%
%% Título e cabeçalho
%\noindent\parbox[c]{.15\textwidth}{\includegraphics[width=.15\textwidth]{logo}}\hfill
\parbox[c]{.825\textwidth}{\raggedright%
  \sffamily {\LARGE

Séries: Notas de Aula

Séries de Taylor e Polinômios de Taylor

\par\bigskip}
{Prof: Felipe Figueiredo\par}
{\url{http://sites.google.com/site/proffelipefigueiredo}\par}
}

Versão: \verb|20151116|

%%%%%%%%%%%%%%%%%%%%%%%%


%%%%%%%%%%%%%%%%%%%%%%%%
\section{Objetivos de aprendizagem}

Ao final desta aula o aluno deve saber \ldots


\section{Pré-requitos da aula}

\begin{itemize}
\item 
\item 
\end{itemize}

\section{Conteúdo}

O aluno deve consultar o livro texto na seção X.Y para se aprofundar
no conteúdo desta aula.

\subsection{Fórmula de Taylor}

Polinômio de Taylor de $f(x)$ de grau $n$ em torno de $x=a$:

\begin{displaymath}
P_n(x) = f(a) + f'(a)(x-a) + \frac{f''(a)}{2!}(x-a)^2 + \frac{f'''(a)}{3!}(x-a)^3 + \ldots + \frac{f^{(n)}(a)}{n!}(x-a)^n
\end{displaymath}

\subsection{Coeficientes do polinômio de Taylor}

Seja $f(x)$ uma função infinitamente derivável no ponto $x=a$.
A série de Taylor é uma maneira de representar a função $f(x)$ como uma série de potências $x^n$, que é uma soma infinita de funções.

Como todos os monômios $x^n$ são os mesmos para qualquer função $f(x)$, o que difere duas funções são os coeficientes destas potências.
A fórmula de Taylor é uma maneira de encontrar os coeficientes que determinam a função de interesse, usando suas derivadas no ponto $x=a$.

{\bf Exemplo:} Calcular o polinômio de Taylor $P_5(x)$ de $f(x)=e^x$, em torno de $x=0$.

Precisamos calcular todas as derivadas até a quinta ($f^{(v)}(x)$).
Como a derivada desta função é igual a ela própria, todas as derivadas são iguais:

\begin{multicols}{2}

Derivadas de $f(x)$:

  \begin{tabular}{rcl}
    $f'(x)$ &=& $e^x$\\
    $f''(x)$ &=& $e^x$\\
    $f'''(x)$ &=& $e^x$\\
    $f^{(iv)}(x)$ &=& $e^x$\\
    $f^{(v)}(x)$ &=& $e^x$\\
  \end{tabular}

  \columnbreak

Valores das derivadas:

  \begin{tabular}{rcl}
    $f(0)$ &= $e^0$ =& 1\\
    $f'(0)$ &=& 1\\
    $f''(0)$ &=& 1\\
    $f'''(0)$ &=& 1\\
    $f^{(iv)}(0)$ &=& 1\\
    $f^{(v)}(0)$ &=& 1\\
  \end{tabular}
\end{multicols}

Agora basta substituir os valores das derivadas na fórmula acima.
Como $a=0$, temos:

$P_5(x) = 1+ 1(x-0) + \frac{1}{2!}(x-0)^2 + \frac{1}{3!}(x-0)^3 + \frac{1}{4!}(x-0)^4 + \frac{1}{5!}(x-0)^5$

Como estamos aproximando $f(x)$ em torno de $x=0$, podemos simplificar a expressão acima e encontrar:

\begin{displaymath}
  P_5(x) = 1+ x + \frac{x^2}{2!} + \frac{x^3}{3!} + \frac{x^4}{4!} + \frac{x^5}{5!}
\end{displaymath}

{\bf Exemplo:} Calcular o polinômio de Taylor $P_7(x)$ de $f(x)=\sin x$ em torno de $x=0$.

Neste exemplo precisamos dos valores das derivadas do seno até a sétima derivada ($f^{(vii)}(x)$).
Felizmente, no caso do seno e do cosseno, não precisamos derivar tantas vezes!

\begin{multicols}{2}

Derivadas de $f(x)$:

  \begin{tabular}{rcl}
    $f'(x)$ &=& $\cos x$\\
    $f''(x)$ &=& $-\sin x$\\
    $f'''(x)$ &=& $-\cos x$\\
    $f^{(iv)}(x)$ &=& $\sin x$\\
    $f^{(v)}(x)$ &=& $\cos x$\\
    $f^{(vi)}(x)$ &=& $-\sin x$\\
    $f^{(vii)}(x)$ &=& $-\cos x$\\
  \end{tabular}

  \columnbreak

Valores das derivadas:

  \begin{tabular}{rcl}
    $f(0)$ &= $\sin 0$ =& 0\\
    $f'(0)$ &=& 1\\
    $f''(0)$ &=& 0\\
    $f'''(0)$ &=& -1\\
    $f^{(iv)}(0)$ &=& 0\\
    $f^{(v)}(0)$ &=& 1\\
    $f^{(vi)}(0)$ &=& 0\\
    $f^{(vii)}(0)$ &=& -1\\
  \end{tabular}
\end{multicols}

Agora basta substituir os valores das derivadas na fórmula acima.
Como $a=0$, temos:

$P_7(x) = 0+ 1(x-0) + \frac{0}{2!}(x-0)^2 + \frac{-1}{3!}(x-0)^3 + \frac{0}{4!}(x-0)^4 + \frac{1}{5!}(x-0)^5 + \frac{0}{6!}(x-0)^6 + \frac{-1}{7!}(x-0)^7$

Como estamos aproximando $f(x)$ em torno de $x=0$, podemos simplificar a expressão acima e encontrar:

\begin{displaymath}
  P_7(x) = x  - \frac{x^3}{3!}  + \frac{x^5}{5!} - \frac{x^7}{7!}
\end{displaymath}

\hrulefill

{\bf Exercício:}

Calcular o polinômio de Taylor $P_8(x)$ de $f(x)=\cos x$ em torno de $x=0$.

Dica: use os mesmos valores do exemplo anterior, mas observe que ordem com que eles aparecem é diferente.

Resposta: $P_8(x) = 1 - \frac{x^2}{2!}  + \frac{x^4}{4!} - \frac{x^6}{6!} + \frac{x^8}{8!}$

\hrulefill

{\bf Exercício:}

Calcular o polinômio de Taylor $P_5(x)$ de $f(x)=e^x$, em torno de $x=2$.

Resposta: $P_5(x) = e^2+ e^2(x-2) + \frac{e^2}{2!}(x-2)^2 + \frac{e^2}{3!}(x-2)^3 + \frac{e^2}{4!}(x-2)^4 + \frac{e^2}{5!}(x-2)^5$

\hrulefill

{\bf Exercício:}

Calcular o polinômio de Taylor $P_5(x)$ de $f(x)=e^{2x}$, em torno de $x=0$.

Resposta: $P_5(x) = 1 + 2x+ \frac{4}{2!}x^2+ \frac{8}{3!}x^3+ \frac{16}{4!}x^4+ \frac{32}{5!}x^5$
\end{document}
