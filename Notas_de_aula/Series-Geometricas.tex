\everymath{\displaystyle}
%\documentclass[pdftex,a4paper]{article}
\documentclass[a4paper]{article}
%%classes: article, report, book, proc, amsproc

%%%%%%%%%%%%%%%%%%%%%%%%
%% Misc
% para acertar os acentos
\usepackage[brazilian]{babel} 
%\usepackage[portuguese]{babel} 
% \usepackage[english]{babel}
% \usepackage[T1]{fontenc}
% \usepackage[latin1]{inputenc}
\usepackage[utf8]{inputenc}
\usepackage{indentfirst}
\usepackage{fullpage}
% \usepackage{graphicx} %See PDF section
\usepackage{multicol}
\setlength{\columnseprule}{0.5pt}
\setlength{\columnsep}{20pt}
%%%%%%%%%%%%%%%%%%%%%%%%
%%%%%%%%%%%%%%%%%%%%%%%%
%% PDF support

\usepackage[pdftex]{color,graphicx}
% %% Hyper-refs
\usepackage[pdftex]{hyperref} % for printing
% \usepackage[pdftex,bookmarks,colorlinks]{hyperref} % for screen

%% \newif\ifPDF
%% \ifx\pdfoutput\undefined\PDFfalse
%% \else\ifnum\pdfoutput > 0\PDFtrue
%%      \else\PDFfalse
%%      \fi
%% \fi

%% \ifPDF
%%   \usepackage[T1]{fontenc}
%%   \usepackage{aeguill}
%%   \usepackage[pdftex]{graphicx,color}
%%   \usepackage[pdftex]{hyperref}
%% \else
%%   \usepackage[T1]{fontenc}
%%   \usepackage[dvips]{graphicx}
%%   \usepackage[dvips]{hyperref}
%% \fi

%%%%%%%%%%%%%%%%%%%%%%%%


%%%%%%%%%%%%%%%%%%%%%%%%
%% Math
\usepackage{amsmath,amsfonts,amssymb}
% para usar R de Real do jeito que o povo gosta
\usepackage{amsfonts} % \mathbb
% para usar as letras frescas como L de Espaco das Transf Lineares
% \usepackage{mathrsfs} % \mathscr

% Oferecer seno e tangente em pt, com os comandos usuais.
\providecommand{\sin}{} \renewcommand{\sin}{\hspace{2pt}\mathrm{sen}}
\providecommand{\tan}{} \renewcommand{\tan}{\hspace{2pt}\mathrm{tg}}

% dt of integrals = \ud t
\newcommand{\ud}{\mathrm{\ d}}
%%%%%%%%%%%%%%%%%%%%%%%%



\begin{document}

%%%%%%%%%%%%%%%%%%%%%%%%
%% Título e cabeçalho
%\noindent\parbox[c]{.15\textwidth}{\includegraphics[width=.15\textwidth]{logo}}\hfill
\parbox[c]{.825\textwidth}{\raggedright%
  \sffamily {\LARGE

Séries: Notas de Aula

Séries Geométricas

\par\bigskip}
{Prof: Felipe Figueiredo\par}
{\url{http://sites.google.com/site/proffelipefigueiredo}\par}
}

Versão: \verb|20151023|

%%%%%%%%%%%%%%%%%%%%%%%%


%%%%%%%%%%%%%%%%%%%%%%%%
\section{Objetivos de aprendizagem}

Ao final desta aula o aluno deve saber reconhecer Séries Geométricas
(tanto somatórios finitos e infinitos), e aplicá-las em problemas com
somas incrementadas com razão constante.


\section{Pré-requitos da aula}

Se $|r|<1$,
\begin{displaymath}
  \lim_{n \rightarrow \infty} r^n = 0
\end{displaymath}

\section{Conteúdo}

O aluno deve consultar o livro texto na seção 9.1 para se aprofundar
no conteúdo desta aula.

\subsection{Formulário}
\label{sec:formulario}

\begin{displaymath}
  S_n = \frac{a(1-r^n)}{1-r}, \textrm{ onde } r \ne 1
\end{displaymath}

\begin{displaymath}
  S_\infty = \frac{a}{1-r}, \textrm{ onde } |r| < 1
\end{displaymath}

\subsection{Problema}

Suponha que você toma um medicamento na forma de cápsula ou
comprimido. Todas as doses contém a mesma quantidade $Q$ de
medicamento, mas conforme o tempo passa, seu corpo vai metabolizando
ou excretando. Assim a quantidade presente no seu corpo vai
diminuindo, e você precisa de uma nova dose, para manter sempre uma
quantidade efetiva de medicamento no organismo. Como podemos descobrir
qual é a quantidade de medicamento {\em presente} no seu corpo em cada
instante de tempo?

Para fixar idéias, considere que você toma um comprimido de 100mg de
medicamentol \copyright\footnote{Medicamento extremamente fictício,
  usado na falta de conhecimento} a cada 8 horas. Vamos assumir que é
conhecido, para este medicamento, que após o período de 8h, restam
apenas $20\%$ da dose de 100mg. Pergunta-se:

\begin{enumerate}
\item Qual é a quantidade $Q_4$ (mg) de medicamento após a ingestão da
  quarta dose? E quantidade $Q_{10}$ ou $Q_{18}$? E $Q_n$?
\item Após um número $n$ muito grande de doses, a quantidade $Q_n$
  continua a crescer indefinidamente, ou estabiliza?
\item Nos casos em que ela estabiliza, o que é necessário para que
  isso ocorra?
\end{enumerate}

\subsection{Séries geométricas}

Considere a seguinte soma infinita:

\begin{displaymath}
  4+ 2+ 1+ \frac{1}{2} +\frac{1}{4} + \frac{1}{8} + \ldots
\end{displaymath}

O que podemos observar sobre cada nova parcela nesta soma? Cada novo
termo é a metade do anterior, certo?

Mas com infinitos termos, precisamos de uma maneira mais compacta para
efetuar essa operação, digamos, em uma calculadora. Precisamos
encontrar uma {\em fórmula fechada}. Como fazer isso?

Primeiro vamos tentar identificar alguma lógica que sirva para todos
os termos. Para isto, basta observar que isto é o mesmo que
multiplicar o termo anterior por $\frac{1}{2}$. Observe:

% \begin{displaymath}
%   4+ 2+ 1+ \frac{1}{2} +\frac{1}{4} + \frac{1}{8} + \ldots
% \end{displaymath}
% \begin{displaymath}
%   4 + 4\times\frac{1}{2} + 4\times\frac{1}{2}^2 + 4\times\frac{1}{2}^3 + 4\times\frac{1}{2}^4 + 4\times\frac{1}{2}^5 + \ldots
% \end{displaymath}

\begin{displaymath}
  \begin{array}{rccccccc}
    &4&+2&+1&+\frac{1}{2}&+\frac{1}{4}&+\frac{1}{8}&+\ldots\\
    &&&&&&&\\
    =&4& + 4\times\left(\frac{1}{2}\right)&+4\times\left(\frac{1}{2}\right)\times\left(\frac{1}{2}\right)&+4\times\left(\frac{1}{2}\right)\times\left(\frac{1}{2}\right)\times\left(\frac{1}{2}\right)& + \ldots& & \\
    &&&&&&&\\
    =&4& +4\times\left(\frac{1}{2}\right)&+4\times\left(\frac{1}{2}\right)^2&+4\times\left(\frac{1}{2}\right)^3&+4\times\left(\frac{1}{2}\right)^4&+4\times\left(\frac{1}{2}\right)^5&+\ldots
  \end{array}
\end{displaymath}
\begin{displaymath}
    = \sum_{n=0}^\infty 4\left(\frac{1}{2}\right)^n
\end{displaymath}

Pense um pouco sobre a fórmula do somatório acima. Substitua $n=0$,
$n=1$, etc, e encontre os primeiros termos para ter certeza que você a
entendeu.

\end{document}
